% !TeX program = xelatex
% !Mode:: "TeX:UTF-8"
%%  本模板推荐以下方式编译: xelatex
%%     1. 文件默认的编码为 UTF-8 对于windows,请选用支持UTF-8编码的编辑器。
%%   2. 若是模板有什么问题,请及时与我们取得联系,Email:latexstudio@qq.com。
%%   3. 可以到  https://ask.latexstudio.net 提问
%%   4. 请安装 最新版本的 TeXLive 地址:
%%   http://mirrors.ctan.org/systems/texlive/Images/texlive.iso

\documentclass[12pt,a4paper]{nmmcm}
\usepackage{ctex}
\usepackage{graphicx}
\usepackage{booktabs,colortbl}
\usepackage{xcolor}
\usepackage{tikz}
\usepackage{indentfirst}
\mcmsetup{CTeX = true,
        tcn ={\xiaowuhao 2024030125260 }, problem = C,
        sheet = true, titleinsheet = false, keywordsinsheet = true,
        titlepage = true, abstract = true}
\usepackage{xurl}
\setmainfont[
    Path=fonts/TimesNewRoman/,
    UprightFont = *-Regular,
    BoldFont = *-Bold,
    ItalicFont = *-Italic,
    BoldItalicFont = *-Bold-Italic  
]{TimesNewRoman}
\setmonofont[ 
    Path=fonts/UbuntuMono/,
    UprightFont = *-Regular,
    BoldFont = *-Bold,
    ItalicFont = *-Italic,
    BoldItalicFont = *-Bold-Italic  
]{UbuntuMono}
\usepackage{lipsum}

\usepackage{paralist}
\let\itemize\compactitem
\let\enditemize\endcompactitem
\let\enumerate\compactenum
\let\endenumerate\endcompactenum
\let\description\compactdesc
\let\enddescription\endcompactdesc

\setlength\abovedisplayskip{5pt}
\setlength\belowdisplayskip{-8pt}
\setlength{\parskip}{0.1em}

\newcommand\wordc[1]{\textbf{#1}}
\renewcommand{\appendixtocname}{附\quad录}
\renewcommand{\appendices}{\hspace{-2em}{\sanhao\HEI {\bf 附~~~录}}}
\colorlet{tableheadcolor}{gray!25} % Table header colour = 25% gray
\newcommand{\headcol}{\rowcolor{tableheadcolor}}

\title{\textcolor{red}{论文的题目(三号黑体)}}
\date{}

\usepackage[font=small,labelfont={bf,sf},tableposition=top]{caption}

\begin{document}
\begin{abstract}
  

%abstract---------------
{\song\xiaosihao
\setlength{\parindent}{2em}\textcolor{red}{(第1段)	问题重述+简要思想:首先简要叙述所给问题的背景和动机,并分别分析每个小问题的特点(以下以三个问题为例)。根据这些特点说出自己的思想:针对于问题1,采用~$\cdots \cdots$ 的方法解决;针对问题2用~$\cdots \cdots$ 的方法解决;针对问题3用~$\cdots \cdots$ 的方法解决。}


\setlength{\parindent}{2em}\textcolor{red}{(第2段)	模型建立及求解结果:介绍思想和模型: 对于问题1我们首先建立了~$\cdots \cdots$ 模型I。首先利用~$\cdots \cdots$ ,其次计算了~$\cdots \cdots$ ,并借助~$\cdots \cdots$ 数学算法和~$\cdots \cdots$ 软件得出了~$\cdots \cdots$ 结论。}

\setlength{\parindent}{2em} \textcolor{red}{(第3段)	对于问题2我们用~$\cdots \cdots$(模型的建立与求解结果的陈述中,思想、模型、软件和结果必须描述清晰,亮点详细说明需突出。}

\setlength{\parindent}{2em}\textcolor{red}{(第4段)	对于问题3我们用~$\cdots \cdots$ (模型的建立与求解结果的陈述中,思想、模型、软件和结果必须描述清晰,亮点详细说明需突出。}

\setlength{\parindent}{2em}\textcolor{red}{(第5段)	优化结果及总结:在~$\cdots \cdots$ 条件下,针对~$\cdots \cdots$ 模型进行适当修改与优化,这种条件的改变可能来自你的一种猜想或建议。要注意合理性。此推广模型可以不深入研究,也可以没有具体结果。}
}

\begin{rmk}
字数300$\sim $600之间,需控制在一页;摘要中必须将具体方法、模型和所得结果写出来;摘要要求“总分总”,段开头可用“针对问题1,针对问题2,针对问题3..”或者“首先,然后,其次,最后”等词语进行有逻辑的论述。摘要是重中之重,必须严格执行!
\end{rmk}







  \begin{keywords}
    {\song\xiaosihao
      \textcolor{red}{天然气水合物、资源预测、核密度估计、克里金插值、目标优化问题}
  \end{keywords}

\end{abstract}
\maketitle
\renewcommand{\contentsname}{\centerline{\sanhao\bfseries\HEI 目\quad 录}}
%\thispagestyle{empty}
%{\song\xiaosihao
\tableofcontents
%}

\newpage
\setcounter{page}{1}
\pagestyle{fancy}
\section{问题重述}
本研究针对天然气水合物(Natural Gas Hydrate/Gas Hydrate),即通常所称的“可燃冰”,
这是一种在特定高压低温环境下与水结合形成的类冰状结晶物质。由于其外观类似冰块且能在火源接触下燃烧,
故此得名。这种物质主要分布在深海沉积物或陆地的永久冻土层中,是一种被国际能源领域高度重视的潜在清洁
能源。
在能源科学研究与勘探领域中,天然气水合物的勘探和量化评估是一项复杂且技术要求高的任务。其关键挑
战包括但不限于:资源的空间定位、量化评估、经济可行性分析以及对气候变化的潜在影响评价。当前,虽
然天然气水合物被视为替代传统石油和天然气资源的重要选择,但相关的资源勘探和评估技术仍显不足,尤
其是在资源量的精确计算和预测方面。
对于天然气水合物资源量的估计主要依赖两种思路:成藏思路和生烃思路。成藏思路方法着眼于天然气水合物的赋存状态,通过确定天然气水合物的聚集区域并评估其规模和数量分布来估计资源量。而生烃思路方法则从有机质的沉积和演化过程出发,通过物质守恒原理模拟水合物的生成和运聚过程。在实际应用中,成藏思路的体积法由于其直观性和适用性较广,成为了评估天然气水合物资源量的主流方法。



\subsection{引言}
%Introduction---------------

\setlength{\parindent}{2em} \textcolor{red}{在保持原题主体思想不变下,可以自己组织词句对问题进行描述,主要数据可以直接复制,对所提出的问题部分基本原样复制。篇幅建议不要超过一页。大部分文字提炼自原题。}


\subsection{要解决的具体问题}
\begin{enumerate}
  \item
        问题1的重述:确定研究区域内天然气水合物的资源分布范围.
  \item
        问题2的重述:评估区域内资源参数如有效厚度、地层孔隙度和饱和度的概率分布及其变化规律。
  \item
        问题3的重述:基于统计分析提出天然气水合物资源量的概率分布估计。
  \item
        问题4的重述:针对资源评估的进一步精细化,提出在研究区域增加钻探井位的策略。
\end{enumerate}


\section{问题分析}
\subsection{问题1的分析}

问题1要求结合附件1,2所给的勘探井位信息,利用数据确定矿井下面天然气水合物资源分布范围。根据天然气水合物静态赋存特征,进行线性计算。研究有助于···
由于数据可能会存在部分异常,这会影响后续过程分析,所以需要对附件一和附件二中的数据进行清洗。
利用清洗后的数据使用线性模型进行计算,分别得出各个点位上天然气水合物含量,从而判断是否属于天然气矿藏,得到分布范围。

\subsection{问题2的分析}
问题2属于$\cdots\cdots$数学问题,对于解决此类问题一般数学方法的分析。
对附件中所给数据特点的分析。
对问题2所要求的结果进行分析。
由于以上原因,我们可以将首先建立一个$\cdots\cdots$的数学模型I,然后将建立一个$\cdots\cdots$的模型II,$\cdots\cdots$对结果分别进行预测,并将结果进行比较.


\subsection{问题3的分析}
问题1属于$\cdots\cdots$数学问题,对于解决此类问题一般数学方法的分析。
对附件中所给数据特点的分析。
对问题1所要求的结果进行分析。
对问题1研究的意义的分析。

\subsection{问题4的分析}
问题1属于$\cdots\cdots$数学问题,对于解决此类问题一般数学方法的分析。
对附件中所给数据特点的分析。
对问题1所要求的结果进行分析。
对问题1研究的意义的分析。


\section{模型假设}
\begin{enumerate}
  \item 假设题目所给的数据真实可靠,无错误记录数据。
  \item 在问题1的计算中,我们假设= A × Z(有效面积×单位厚度)计算标准为单位体积。
  \item 假设具有天然气特性的物质排除天然气以外的任何物质,地壳内不存在其他混淆干扰物,天然气水合物在研究区域内呈连续分布,没有断层或其他地质障碍导致的断裂。
  \item 假设观测期内储层参数保持稳定,研究区域内的地质结构相对均质,在勘探和开发周期内,天然气水合物的赋存状态不会因外界因素(如温度、压力变化)而发生显著变化。不受其他因素影响
  \item 在处理孔隙度、饱和度等参数的统计数据时,假定各钻孔点的数据相互独立。这意味着每个数据点提供的信息是独立的,没有因地理位置或地质联系而产生的相关性。
\end{enumerate}

\section{名词解释与符号说明}

\subsection{名词解释与说明}
\begin{enumerate}
  \item \wordc{有效厚度:}是指储油层中具有工业产油能力的那部分油层的厚
        度,即工业油井内具可动油的储集层的厚度。有非 0 饱和度的点存在
        天然气水合物。
  \item \wordc{水合物饱和度:}在具体条件下,通过修正系数对理论通行能力修正后得到的单
        位时间内所能通过的最大交通量。
  \item \wordc{产气量因子:}指 1 立方米天然气水合物能量相当于多少立方
        米的天然气的热值。

\end{enumerate}

\subsection{主要符号与说明}

\begin{table}[h!]
  \centering
  \small
  \begin{tabular}{p{60pt}<{\centering}|p{60pt}<{\centering}p{180pt}<{\raggedright}}
    \hline
    \headcol 序号 & 符号           & 符号说明(定义)           \\
    \hline
    1           & $ Q $        & 天然气水合物总资源量         \\
    2           & $A$   & 有效面积         \\
    3           & $Z$  & 有效厚度          \\
    4           & $\varPhi $  & 孔隙度          \\
    5           & $S$ & 水合物饱和度         \\
    6           & $E$   & 产气量因子 \\
    7           & $ X_{i} $  &第 i 个钻探点的横坐标             \\
    8           & $ Y_{i}$ & 第 i 个钻探点的纵坐标           \\
    9           & $Z_{i}$ & 车道宽度和侧向净宽修正系数      \\
    10          & $\varPhi_{i}$ & 第 i 个钻探点的深度            \\
    11          & $S_{i}$ & 第 i 个钻探点的孔隙度        \\
    12          & $K_{j}$      & 第 i 个钻探点的水合物饱和度             \\
    \hline
  \end{tabular}
  %\caption{符号与说明}
  \label{symbol}
\end{table}

\section{模型的建立与求解}

数据的预处理:
根据题目需求描述,我们需要对每个井位下的天然气水合物含量、有效厚度、饱和度、概率分布等变化规律进行分析与建模,附件一和附件二提供了钻井测量数据,井位信息,但数据存在部分误差,以及失访情况下造成的数据缺失,需要对原始数据进一步处理方可使用。
\begin{enumerate}
  \item 格式标准化:为确保数据处理的一致性和后续分析的顺利进行,我们手动对所提供数据的分隔符进行了统一。具体操作主要有三:将数据中的分隔符不统一问题统一转换为制表符;将原始数据文件的编码格式由GDK转换为UTF-8编码防止出现读取乱码问题;所有文件按照英文命名规则进行重命名。
  \item 读取所有数据,用空值替换掉缺失值。
  \item 异常值剔除
        孔隙度是指岩石中孔隙空间的体积与岩石总体积的比值,它是衡量岩石多孔性的一个重要参数,通常用小数或百分比表示。含水合物饱和度是指在岩石孔隙空间中,天然气水合物所占的体积与孔隙总体积的比值。在附件一中,孔隙度和含水合物饱和度出现了理论范围内的正常值和负值。由于负值是异常,
        基于负值的计算会使得资源量出现负值,但是考虑到数据具体测量方式未知,试探性使用 Min Max 标准化方法,将数据序列按照大小映射到 [0,1] 区间
        %公式:处理结果/图:
        我们发现,尽管该方法能够消除负值,但同时也导致了数据的过度失真。标准化后的数据丧失了原有的地质学意义,放弃采用Min-Max标准化方法。
        最后对原始数据集中出现的负数值全部进行了识别和排除。
        %处理后的数据如下:
\end{enumerate}
\subsection{问题1的分析和求解}
\subsubsection{问题1模型的建立}
我们需要解决的问题是根据附件中勘探井位信息,
确定天然气水合物资源分布范围,并给出详细数据和图表,在后续分析更具体的分布奠定基础。通过对附件一和附件二中数据的观察,钻井测量数据不仅包含了深度,还包含了孔隙度和含水合物饱和度。天然气水合物的储层参数主要包括水合物的饱和度、分布深度、分布面积、孔隙度、渗透率等,而资源量的评估更是受到了水合物饱和度、分布深度、分布面积和孔隙度的影响。基于成藏思路的方法从本质上来讲是体积法,
体积法能反映资源的实际状态,便于指导实际开发选址,因此是体积法最常用的水合物资源量估计方法。


\subsubsection{问题2模型的求解}
将预处理数据带入上述资源量与储层参数的线性关系,计算出天然气水合物资源量Q,通过可视化处理得到天然气水合物资源的三维深度和资源量总分布图,以及各井口内部天然气水合物资源深度与资源量的对应关系。

\begin{figure}[h!t]
  \centerline{\includegraphics[scale=1]{task1-1.pdf}}
  \caption{\song\wuhao  天然气水合物资源的三维深度和资源量总分布}
\end{figure}

\begin{figure}[h!t]
  \centerline{\includegraphics[scale=1]{task1-2.pdf}}
  \caption{\song\wuhao 各井口内资源深度与资源量对应关系}
\end{figure}

\begin{figure}[h!t]
  \centerline{\includegraphics[scale=1]{task1-3.pdf}}
  \caption{\song\wuhao 井间资源分配}
\end{figure}
 
\begin{align}
  A_{\max} & =\dfrac{3600}{t_{\min}}=\dfrac{3600}{J_{\min} /(v / 3.6)}
  =\dfrac{1000 v}{J_{\min }}(\text{辆 } / h)                            \\
  J_{\min} & =J_{\rm r}+J_{z}+J_{\rm a}
\end{align}


\subsubsection{问题2结果}




\subsection{问题 三的求解和分析 的求解和分析 的求解和分析}

\subsubsection{对问题的分析}

问题 三要求我们 $\cdots$。

\subsubsection{对问题的求解}

\textbf{模型 Ⅱ—基于 负荷度 负荷度 分析 的小区开放影响度综合评价}

(1)模型的准备

1)负荷度介绍

负荷度( V/CV/CV/C)是指在理想条件下,最大服务交通量与基本行能力之比.

2)数据处理

将道路分为主干和次,其要参数详见 表 10

\begin{table*}[h!]
  \centering
  \small
  \tabcolsep 2.5pt
  \caption{主次道路参数表}
  \begin{tabular*}{0.8\linewidth}{p{60pt}<{\centering}p{60pt}<{\centering}
    p{60pt}<{\centering}p{80pt}<{\centering}p{80pt}<{\centering}}
    \toprule
    道路类型  &  主干路  &  支干路  &  小区内宽道路  &  小区内窄道路  \\
    \midrule
    行车速度  & 50 km / h & 40 km / h & 30 km / h & 20 km / h \\
    车道数  & 4 & 3 & 2 & 1 \\
    \bottomrule
  \end{tabular*}
  \label{tab10}
\end{table*}

(2)模型的建立

1)小区的分类

根据小区结构,周边道路分布形状和周边道路车道数的不同,我们将小区分
别分为~4、2、3 类,小区的分类结果详见表~11


2)计算周边各路段及交叉口的通行能力


对于周边各路段的通行能力,我们运用问题二已建立的模型进行计算.在此
基础上对于交叉口的通行能力交叉口~G 我们建立公式如下:

\begin{align}
  G_{\text{交又口}} & =\sum_{i=1}^{n} G_{i} \\
  G_{i}          & =\sum_{j=1}^{k} C_{j}
\end{align}


其中,$C_{j}$ 为进口各车道的通行能力,$ G_{i}$ 为交叉口各进口的通行能力.


3)建立影响度综合评价体系~[9][10][11]

我们采用先单项评价再综合评价的方法,其总体思路见表~12

\begin{table*}[h!]
  \centering
  \small
  \tabcolsep 2.5pt
  \caption{小区分类表}
  \begin{tabular*}{0.8\linewidth}{p{100pt}<{\centering}|p{60pt}<{\raggedright}|p{180pt}<{\raggedright}}
    \hline
    分类标准 & 类型名称& 类型说明\\
    \hline
    \multirow{4}*{小区结构 }& A组团有序型 & 小区楼房呈组团型分布,每一区域间隔较大,开放后小区
    道路较宽,且区域间分布有序\\

    & B紧凑有序型 & 小区楼房间隔紧凑,且排列有序,开放后道路网格呈“街
    区型”,特点为“高密度、窄路宽.\\
    &C组团无序型& 小区楼房呈组团式分布,每一区域间隔较大,开放后小区
    道路较宽,但区域间分布杂乱小区楼房间隔紧凑,但排列杂乱,开放后小区道路呈现“低\\
    &D紧凑无序型&密度,窄路宽”的特点\\

    \multirow{2}*{周边道路形状分布}& 四周围绕型&四周均为道路\\

    &半边包围型&半边围绕道路\\

    \multirow{3}*{车道数(针对半封闭性)}& 主干道型 & 两条道路均为主干道\\

    &次干道型 & 两条道路均为次干道\\

    &混合型& 两条道路一主一次\\
    \hline
  \end{tabular*}
  \label{tab11}
\end{table*}

\begin{table*}[h!]
  \centering
  \small
  \tabcolsep 2.5pt
  \caption{综合评价思路表}
  \begin{tabular*}{0.8\linewidth}{p{100pt}<{\centering}|p{160pt}<{\raggedright}|p{80pt}<{\raggedright}}
    \hline
    评价性质  &  评价内容  &  评价指标  \\
    \hline
    \multirow{2}*{ 单项评价 } & \multirow{2}*{  局部路段及交叉口交通负荷影响 } &  路段影响度  \\
    & &交叉口影响\\
    \multirow{2}*{ 综合评价 } & \multirow{2}*{整个路网交通负荷影响} &平均路段影响度  \\
    &&平均交叉口影响度\\
    \hline
  \end{tabular*}
  \label{tab12}
\end{table*}

A. 负荷度单项评价

a. 封闭式小区开放后,新增小区内道路对于周边某一路段i 的影响度 $K_{si}$
根据公式计算:
\begin{align}
  K_{s i}   & =\dfrac{I_{s i p}-I_{s i b}}{B_{s i}} \\
  I_{s i p} & =I_{s i b}+a
\end{align}

其中,$I _{sip}$ 为小区道路建成后路段 i 上高峰小时交通量,$I _{sib}$ 为不考虑小区道
路建成后新增交通量的情况下,路段 i 的高峰小时交通量,  $B_{s i}$  为路段 $i$ 的设计
通行能力,$a$ 为开放后小区道路的通行量.
b. 封闭式小区开放后,新增小区内道路对于周边道路交叉口的影响度  $K_{c i}$
根据公式计算:
\begin{align}
  K_{c i}=\dfrac{I_{c i p}-I_{c b}}{B_{c t}}
\end{align}


其中,$K_a$ 为小区道路建成后对交叉口 i 的影响度,$I_{crp}$ 为小区道路建成后交 叉口 $i$
上高峰小时交通量, $ I_{c i b}$  为不考虑小区道路建成后新增交通量的情况下, 交叉口 i 的
高峰小时交通量,  $B_{c i}$  为交叉口 $i$ 的设计通行能力.


\begin{figure}[h!t]
  \centerline{\includegraphics[scale=1]{fig1.pdf}}
  \caption{\song\wuhao 图~3的标题名称}
\end{figure}


\section{模型的评价与推广 模型的评价与推广}

\textcolor{red}{将模型进行数值计算,并与附件中的真实采样值(进行列表或图示)比较。对误差进行数据分析,给出误差分析的理论估计。}

\subsection{模型的评价}


1. 优点

\textcolor{red}{得到满意的解、
  较好地解决了$\cdots$问题、
  使模型得到简化、
  使结果更合理,避免…带来的较大误差、
  使问题描述比较清晰、
  减少大的计算量。
}

(1)问题求解中 辅之流程图, 将建模思路完整清晰的展现出来;

(2)问题二在对 问题二在对理论通行能力进修复时考虑因素 细致、全面,理论通行能力进修复时考虑因素
细致、全面,系数准确度高;

(3)在问题三中,提出“影响度”的概念较为直观地定量给小区开放后的效果,简便有.在影响度计算上由
点及面从每个路段、交叉口到整 个路网,层深入具有逻辑性;

\begin{figure}[h!t]
  \centerline{\includegraphics[scale=1]{fig4.pdf}}
  \caption{\song\wuhao 图~3的标题名称}
\end{figure}


(4)运用多种数学软件(如 MATLAB、SPSS),取长补短,使计算结果更加),取长补短,使计算结果更
加 准确、明晰.

2. 缺点

\textcolor{red}{主观性过强、
  建立在什么的前提条件下、
  有一定的局限性、
  存在不确定性、
  有一定的偏差。
}

(1)在数学软件的计算中会将小数计算 结果进行保留,使得随后的会将小数计算 结果进行保留,使得随后
的或统计结果造成一定误差;

(2)问题二求解修正通行能力时多次使用了查表,操作不够简便.

\subsection{模型的、模型的 推广}

\begin{itemize}

  \item \textcolor{red}{对本文中的模型给出比较客观的评价,必须实事求是,有根据,以便评卷人参考。}

  \item \textcolor{red}{推广和优化,需要花费功夫想出合理的、甚至可以合理改变题目给出的条件的、不一定可行但是具有一定想象空间的准理想的方法、模型。由此做出一些改进方向,也可以是参赛者一些来不及实现的思路。}
\end{itemize}

1. 问题二中 建立 的模型 在现实 生活 中可以 作为 检验 数据 对实测数据 的准确 性进行 检验,帮助 人们
更好 的测算 交通 数据.

2. 基于问题三建立的模型,可以根据道路实时检测数(某段单位间内 基于问题三建立的模型,推算新建
一条道路对于当前交 通状况的改善效果,帮助度等).

\section{模型的改进}

\subsection{模型一的改进}
针对问题二中的模型一,在具体求解大型车对车辆通行能力的修正系数时,
我们利用交通量的测算值对照得到相应的大型车修正系数.但是,在实际操作中
交通量的测定有很大的难度,如果此时交通量数据无法得到,那么我们便不能得
到相应的修正系数,因此我们对模型进行改进.

由~GREENSHIELD K-V 线性模型,可得通行能力的公式:
\begin{align}
  A_{p}=\begin{cases}
          \dfrac{3600}{t}\left(1-\dfrac{3.6 l}{V_{t} t}\right)\left(V_{f}>7.2 l / t\right) \\
          \dfrac{250 V_{f}}{t}\left(V_{f} \leq 7.2 l / t\right)
        \end{cases}
\end{align}

对应的临界车辆速度:
\begin{align}
  V_{p}=\begin{cases}
          \dfrac{V_{f}-3.6 l}{t} & \left(V_{f}>7.2 l / t\right)      \\
          \dfrac{1}{2} V_{f}     & \left(V_{f} \leq 7.2 l / t\right)
        \end{cases}
\end{align}

由美国道路通行能力准则可得,美国将道路服务水平分为六级:A-F 级,而
我国目前针对当前国情,将道路服务水平分成四级:一级相当于美国的A、B 两
级;二级相当于美国的C 级;三级相当于美国的D 级;四级相当于美国的E、F
级。因此,相应的,将美国服务水平划分标准进行针对性修正,得到中国道路服
务水平划分标准,见表

\begin{table*}[h!]
  \centering
  \small
  \tabcolsep 2pt
  \caption{我国服务水平划分标准}
  \begin{tabular*}{0.87\linewidth}{p{60pt}<{\centering}p{40pt}<{\centering}
    p{40pt}<{\centering}p{40pt}<{\centering}p{40pt}<{\centering}
    p{80pt}<{\centering}p{40pt}<{\centering}}
    \toprule
    服务水平 (L0S)  & \multicolumn{2}{c} {一级 } & 二级  & 三级  & \multicolumn{2}{c} {四级 } \\
    \cline{2-3}\cline{6-7}
    服务交通量  & 800 & 1200 & 1800 & 2500 & $A_{D}$ & $\leqslant A_{P}$ \\
    速度  km / h & 120 & 120 & 120 & 120 & $\geqslant V_{p}$ & $\leqslant V_{p}$ \\
    V / C & 0.33 & 0.48 & 0.71 & 1.0 & $A_{p} / A_{\max}\leqslant 1.0$ & -(无意义 ) \\
    \bottomrule
  \end{tabular*}
\end{table*}

由于车流量的测算相对于交通量来说较易得到,我们便可以不用对交通量进
行测算,可以通过车流量与通行能力的比值计算出~V/C 饱和度值,再通过该值对
照我国服务水平划分标准,间接得到服务交通量,从而得到大型车对通行能力的
修正系数.


\subsection{模型二的改进}

针对于问题三中的模型,在得出各个类型小区在开放后对于整个小区周边路
网交通负荷影响度后,无法判别小区开放的效果是积极的还是消极的,由此我们
可以采用~Bress 悖论的原理进行判别:在个人独立选择路径的情况下,为某路网
增加额外的通行能力(如增加路段的等),反而会导致整个路网的整体运行水平
降低的情况.

将路网进行简化如图~15:

根据推导可得: 当 $\beta_{3}/\left(\beta_{1}+\beta_{2}\right) \leq\left(\beta_{5}+\beta_{6}\right)/\beta_{4}$ 时,会发生悖论,即道路的开
通反而会加剧原有道路的交通状况.

\textcolor{red}{需重新起页,不得与论文正文内容在同一页上}

\begin{rmk}
  5篇以上!
\end{rmk}

\newpage

\begin{thebibliography}{99}
  \addcontentsline{toc}{section}{参考文献}
  \bibitem{1} 张金川.页岩气成藏机理和分布[J].天然气工业,2004,(7):15-18.

  \bibitem{2} 粟科华.天然气水合物三维成藏物模实验系统的构建与检验[J].天然气工业,2013,(12):173-178.
  \bibitem{3} 梁金强.天然气水合物资源量估算方法及应用[J].地质通报,2006,(9):1205-1210.
  \bibitem{4} 黄杰.基于核密度估计的基本概率指派生成方法[J].计算机应用研究,2020,(7):2037-2040,2044.
  \bibitem{5} 【前沿论坛】庞雄奇:深层致密介质中油气富集成藏动力机制与演化模式
  \url{http://www.igg.cas.cn/xwzx/cutting_edge/202107/t20210728_6149403.html}
  \bibitem{6}Cressie N. The origins of kriging[J]. Mathematical geology, 1990, 22: 239-252.
  \bibitem{7}Li X J, Wang C H, Zhu H H. Kriging interpolation and its application to generating stratum model[J]. Rock and soil mechanics, 2009, 30(1): 157-162.
  \bibitem{8}Oliver M A, Webster R. Kriging: a method of interpolation for geographical information systems[J]. International Journal of Geographical Information System, 1990, 4(3): 313-332.
  \bibitem{9}

\end{thebibliography}
\newpage

\begin{appendices}

  \section*{问题 1 的相关数据与程序}

  \textbf{{0.98,0.00,0.00}{程序一:MATLAB算道路车辆通行能力:}}
  \lstinputlisting[language=Matlab]{./code/mcmthesis-matlab1.m}

  \section*{}

  \textcolor[rgb]{0.98,0.00,0.00}{\textbf{程序二:C++ 求解路网正体影响度:}}
  \lstinputlisting[language=C++]{./code/mcmthesis-sudoku.cpp}

  \newpage
  \def\thesection{A}
  \renewcommand{\thetable}{\wuhao A-\arabic{table}}
  \setcounter{table}{0}
  \section*{数据表格}
  \textcolor[rgb]{0.98,0.00,0.00}{\textbf{表格数据:}}
  
\begin{table*}[h!]
  \centering
  \small
  \caption{附表1数据}
  \begin{tabular*}{\linewidth}{p{40pt}<{\centering}p{30pt}<{\centering}
    p{30pt}<{\centering}p{40pt}<{\centering}p{50pt}<{\centering}p{70pt}<{\centering}
    p{60pt}<{\centering}p{50pt}<{\centering}}
    \toprule
    样本编号 &  车速 & 车道数  & 侧向 净宽 &  车道宽  &  司机反应时间  & 车辆南止耗时  &  交通量  \\
    \midrule
    1 & 37 {\color{Blue} } & 2 & 1 & 3 & 0.5 & 1.72 & 1112 \\
    2 & 47 & 3 & 2.5 & 3.5 & 0.6 & 2.41 & 1835 \\
    3 & 48 & 3 & 2.5 & 3.25 & 1.2 & 2.475 & 2034 \\
    4 & 44 & 2 & 2.5 & 3.25 & 1 & 2.26 & 1477 \\
    5 & 46 & 3 & 2.5 & 3 & 1.2 & 2.27 & 1648 \\
    6 & 53 & 2 & 2.5 & 3.5 & 1.2 & 2.498 & 1952 \\
    7 & 54 & 3 & 3.5 & 3.5 & 2 & 2.5 & 2249 \\
    8 & 59 & 3 & 3.5 & 3.5 & 0.7 & 2.634 & 1893 \\
    9 & 59 & 3 & 3.5 & 3.25 & 0.2 & 2.642 & 2245 \\
    10 & 48 & 3 & 2.5 & 3.25 & 0.3 & 2.46 & 2035 \\
    11 & 50 & 3 & 4.5 & 3.5 & 0.3 & 2.52 & 2318 \\
    12 & 56 & 3 & 3.5 & 3.25 & 0.9 & 2.617 & 2203 \\
    13 & 57 & 2 & 2.5 & 3.5 & 0.8 & 2.625 & 2034 \\
    14 & 58 & 2 & 2.5 & 3 & 0.6 & 2.641 & 2178 \\
    15 & 68 & 4 & 3.5 & 3.25 & 0.9 & 3.05 & 2468 \\
    16 & 59 & 3 & 4.5 & 3.75 & 0.6 & 2.975 & 2406 \\
    17 & 75 & 4 & 4.5 & 3.75 & 0.7 & 3.15 & 2648 \\
    18 & 22 & 1 & 1 & 3 & 1.1 & 1.45 & 800 \\
    19 & 27 & 4 & 0.5 & 3 & 0.6 & 1.5 & 903 \\
    20 & 75 & 1 & 2.5 & 3.5 & 0.6 & 1.46 & 1010 \\
    21 & 76 & 1 & 3.5 & 3.5 & 0.2 & 1.63 & 1069 \\
    22 & 46 & 2 & 1.5 & 3.25 & 1.9 & 2.3 & 1682 \\
    23 & 46 & 2 & 2.5 & 3.25 & 1 & 2.32 & 1734 \\
    24 & 46 & 2 & 2.5 & 3.75 & 0.2 & 2.4 & 1826 \\
    25 & 47 & 3 & 2.5 & 3.25 & 1.2 & 2.37 & 1625 \\
    26 & 77 & 3 & 4.5 & 3.5 & 0.2 & 2.475 & 2148 \\
    27 & 48 & 3 & 4.5 & 3.25 & 0.3 & 2.47 & 2278 \\
    28 & 80 & 3 & 2.5 & 3.5 & 0.5 & 2.58 & 2177 \\
    29 & 66 & 2 & 3.5 & 3.5 & 1 & 2.72 & 2249 \\
    30 & 67 & 4 & 3.5 & 3.75 & 0.9 & 2.975 & 2484 \\
    31 & 25 & 3 & 1.5 & 3.5 & 0.6 & 1.3 & 846 \\
    32 & 34 & 2 & 4.5 & 3.5 & 0.8 & 1.52 & 1152 \\
    33 & 47 & 3 & 2.5 & 3.25 & 0.3 & 2.42 & 1753 \\
    34 & 48 & 4 & 2.5 & 3.75 & 0.3 & 2.34 & 1924 \\
    35 & 79 & 3 & 2.5 & 3.25 & 1.1 & 2.53 & 2159 \\
    36 & 55 & 3 & 0.5 & 3.5 & 0.9 & 2.62 & 1568 \\
    37 & 78 & 2 & 1 & 3.5 & 0.9 & 2.618 & 2148 \\
    38 & 59 & 3 & 1 & 3.5 & 1 & 2.64 & 2272 \\
    39 & 19 & 1 & 0 & 3 & 1.2 & 1.4 & 513 \\
    40 & 19 & 2 & 2.5 & 3.25 & 1 & 1.35 & 810 \\
    41 & 37 & 2 & 2.5 & 3 & 1.2 & 1.49 & 1102 \\
    42 & 45 & 2 & 2.5 & 3.5 & 0.8 & 2.28 & 1525 \\
    \bottomrule
  \end{tabular*}
  \label{Ap1}
\end{table*}

\newpage

\begin{table*}[h!]
  \centering
  \small
  \caption{小区开放前VISSIM正常行驶仿真数据记录表1}
  \begin{tabular*}{\linewidth}{p{40pt}<{\centering}p{30pt}<{\centering}
    p{30pt}<{\centering}p{40pt}<{\centering}p{50pt}<{\centering}p{70pt}<{\centering}
    p{60pt}<{\centering}p{50pt}<{\centering}}
    \toprule
    样本编号 &  车速 & 车道数  & 侧向 净宽 &  车道宽  &  司机反应时间  & 车辆南止耗时  &  交通量  \\
    \midrule
    47 & 67 & 1 & 0.5 & 3.75 & 0.2 & 2.83 & 2249 \\
    48 & 67 & 4 & 3.5 & 3.25 & 0.6 & 2.815 & 2463 \\
    49 & 75 & 2 & 3.5 & 3.5 & 0.6 & 3.21 & 2748 \\
    50 & 34 & 2 & 1.5 & 3 & 1 & 1.48 & 957 \\
    51 & 39 & 2 & 2.5 & 3.5 & 0.8 & 1.97 & 1364 \\
    52 & 40 & 3 & 2.5 & 3.25 & 0.5 & 2 & 1359 \\
    53 & 50 & 3 & 2.5 & 3.5 & 1 & 2.51 & 2264 \\
    54 & 55 & 2 & 3.5 & 3.25 & 1.2 & 2.6 & 1978 \\
    55 & 55 & 3 & 3.5 & 3.5 & 0.6 & 2.61 & 2218 \\
    56 & 59 & 3 & 0.5 & 3 & 0.2 & 2.638 & 1974 \\
    57 & 63 & 4 & 2.5 & 3.5 & 1.1 & 2.78 & 2384 \\
    58 & 67 & 3 & 2.5 & 3.75 & 0.8 & 2.83 & 2384 \\
    59 & 75 & 3 & 4.5 & 3.5 & 0.3 & 3.2 & 2801 \\
    60 & 77 & 2 & 4.5 & 3.5 & 0.2 & 3.18 & 2845 \\
    61 & 23 & 1 & 0 & 3 & 0.5 & 1.44 & 458 \\
    62 & 75 & 2 & 1 & 3 & 0.2 & 1.625 & 1065 \\
    63 & 46 & 2 & 2.5 & 3.5 & 1 & 2.43 & 1752 \\
    64 & 61 & 2 & 0.5 & 3 & 1.2 & 2.71 & 1890 \\
    65 & 36 & 3 & 2.5 & 3.5 & 1 & 1.67 & 1233 \\
    66 & 38 & 2 & 3.5 & 3 & 1.7 & 1.9 & 1246 \\
    67 & 55 & 1 & 0.5 & 3.5 & 0.3 & 2.615 & 1763 \\
    68 & 74 & 2 & 1.5 & 3.75 & 0.7 & 3.05 & 2349 \\
    69 & 79 & 4 & 2.5 & 3.75 & 0.4 & 3.17 & 2694 \\
    70 & 38 & 2 & 3.5 & 3 & 1.1 & 1.86 & 1343 \\
    71 & 61 & 3 & 1.5 & 3.25 & 0.3 & 2.68 & 2006 \\
    72 & 79 & 3 & 3.5 & 3.5 & 2.1 & 3.48 & 2948 \\
    73 & 27 & 2 & 1 & 3.75 & 0.8 & 1.48 & 928 \\
    74 & 28 & 1 & 1 & 3 & 0.9 & 1.47 & 947 \\
    75 & 34 & 2 & 1 & 3 & 0.3 & 1.49 & 998 \\
    76 & 44 & 3 & 2.5 & 3.25 & 0.3 & 2.24 & 1520 \\
    77 & 78 & 3 & 4.5 & 3.5 & 0.7 & 3.09 & 2648 \\
    78 & 73 & 3 & 3.5 & 3.5 & 1.2 & 3.19 & 2741 \\
    80 & 37 & 4 & 1 & 3 & 1.7 & 1.87 & 1265 \\
    81 & 37 & 2 & 3.5 & 3.5 & 1.5 & 1.84 & 1325 \\
    82 & 38 & 2 & 2.5 & 3 & 1.2 & 1.95 & 1233 \\
    83 & 38 & 2 & 1 & 3 & 2.1 & 1.97 & 1249 \\
    84 & 40 & 2 & 1.5 & 3 & 0.4 & 2.12 & 1366 \\
    85 & 42 & 3 & 4.5 & 3.75 & 0.4 & 2.16 & 1638 \\
    86 & 40 & 3 & 1.5 & 3.25 & 0.8 & 2.43 & 1384 \\
    87 & 41 & 3 & 1.5 & 3.5 & 1.1 & 2.05 & 1434 \\
    88 & 78 & 3 & 4.5 & 3.75 & 1.4 & 3.42 & 3048 \\
    89 & 41 & 4 & 1.5 & 3.75 & 0.8 & 2.15 & 1566 \\
    90 & 42 & 2 & 4.5 & 3.25 & 1.8 & 2.2 & 1466 \\
    91 & 44 & 2 & 2.5 & 3 & 1.2 & 2.24 & 1475 \\
    92 & 75 & 4 & 4.5 & 3.75 & 1.8 & 3.25 & 2801 \\
    93 & 37 & 4 & 4.5 & 3.75 & 0.2 & 1.5 & 2043 \\
    94 & 37 & 2 & 1 & 3 & 2.1 & 1.89 & 1289 \\
    95 & 40 & 4 & 2.5 & 3.5 & 0.9 & 2.15 & 1406 \\
    \bottomrule
  \end{tabular*}
  \label{Ap2}
\end{table*}



\begin{table*}[h!]
  \centering
  \small
  \caption{小区开放前VISSIM正常行驶仿真数据记录表2}
  \begin{tabular*}{\linewidth}{p{40pt}<{\centering}p{30pt}<{\centering}
    p{30pt}<{\centering}p{40pt}<{\centering}p{50pt}<{\centering}p{70pt}<{\centering}
    p{60pt}<{\centering}p{50pt}<{\centering}}
    \toprule
    样本编号 &  车速 & 车道数  & 侧向 净宽 &  车道宽  &  司机反应时间  & 车辆南止耗时  &  交通量  \\
    \midrule
    47 & 67 & 1 & 0.5 & 3.75 & 0.2 & 2.83 & 2249 \\
    48 & 67 & 4 & 3.5 & 3.25 & 0.6 & 2.815 & 2463 \\
    49 & 75 & 2 & 3.5 & 3.5 & 0.6 & 3.21 & 2748 \\
    50 & 34 & 2 & 1.5 & 3 & 1 & 1.48 & 957 \\
    51 & 39 & 2 & 2.5 & 3.5 & 0.8 & 1.97 & 1364 \\
    52 & 40 & 3 & 2.5 & 3.25 & 0.5 & 2 & 1359 \\
    53 & 50 & 3 & 2.5 & 3.5 & 1 & 2.51 & 2264 \\
    54 & 55 & 2 & 3.5 & 3.25 & 1.2 & 2.6 & 1978 \\
    55 & 55 & 3 & 3.5 & 3.5 & 0.6 & 2.61 & 2218 \\
    56 & 59 & 3 & 0.5 & 3 & 0.2 & 2.638 & 1974 \\
    57 & 63 & 4 & 2.5 & 3.5 & 1.1 & 2.78 & 2384 \\
    58 & 67 & 3 & 2.5 & 3.75 & 0.8 & 2.83 & 2384 \\
    59 & 75 & 3 & 4.5 & 3.5 & 0.3 & 3.2 & 2801 \\
    60 & 77 & 2 & 4.5 & 3.5 & 0.2 & 3.18 & 2845 \\
    61 & 23 & 1 & 0 & 3 & 0.5 & 1.44 & 458 \\
    62 & 75 & 2 & 1 & 3 & 0.2 & 1.625 & 1065 \\
    63 & 46 & 2 & 2.5 & 3.5 & 1 & 2.43 & 1752 \\
    64 & 61 & 2 & 0.5 & 3 & 1.2 & 2.71 & 1890 \\
    65 & 36 & 3 & 2.5 & 3.5 & 1 & 1.67 & 1233 \\
    66 & 38 & 2 & 3.5 & 3 & 1.7 & 1.9 & 1246 \\
    67 & 55 & 1 & 0.5 & 3.5 & 0.3 & 2.615 & 1763 \\
    68 & 74 & 2 & 1.5 & 3.75 & 0.7 & 3.05 & 2349 \\
    69 & 79 & 4 & 2.5 & 3.75 & 0.4 & 3.17 & 2694 \\
    70 & 38 & 2 & 3.5 & 3 & 1.1 & 1.86 & 1343 \\
    71 & 61 & 3 & 1.5 & 3.25 & 0.3 & 2.68 & 2006 \\
    72 & 79 & 3 & 3.5 & 3.5 & 2.1 & 3.48 & 2948 \\
    73 & 27 & 2 & 1 & 3.75 & 0.8 & 1.48 & 928 \\
    74 & 28 & 1 & 1 & 3 & 0.9 & 1.47 & 947 \\
    75 & 34 & 2 & 1 & 3 & 0.3 & 1.49 & 998 \\
    76 & 44 & 3 & 2.5 & 3.25 & 0.3 & 2.24 & 1520 \\
    77 & 78 & 3 & 4.5 & 3.5 & 0.7 & 3.09 & 2648 \\
    78 & 73 & 3 & 3.5 & 3.5 & 1.2 & 3.19 & 2741 \\
    80 & 37 & 4 & 1 & 3 & 1.7 & 1.87 & 1265 \\
    81 & 37 & 2 & 3.5 & 3.5 & 1.5 & 1.84 & 1325 \\
    82 & 38 & 2 & 2.5 & 3 & 1.2 & 1.95 & 1233 \\
    83 & 38 & 2 & 1 & 3 & 2.1 & 1.97 & 1249 \\
    84 & 40 & 2 & 1.5 & 3 & 0.4 & 2.12 & 1366 \\
    85 & 42 & 3 & 4.5 & 3.75 & 0.4 & 2.16 & 1638 \\
    86 & 40 & 3 & 1.5 & 3.25 & 0.8 & 2.43 & 1384 \\
    87 & 41 & 3 & 1.5 & 3.5 & 1.1 & 2.05 & 1434 \\
    88 & 78 & 3 & 4.5 & 3.75 & 1.4 & 3.42 & 3048 \\
    89 & 41 & 4 & 1.5 & 3.75 & 0.8 & 2.15 & 1566 \\
    90 & 42 & 2 & 4.5 & 3.25 & 1.8 & 2.2 & 1466 \\
    91 & 44 & 2 & 2.5 & 3 & 1.2 & 2.24 & 1475 \\
    92 & 75 & 4 & 4.5 & 3.75 & 1.8 & 3.25 & 2801 \\
    93 & 37 & 4 & 4.5 & 3.75 & 0.2 & 1.5 & 2043 \\
    94 & 37 & 2 & 1 & 3 & 2.1 & 1.89 & 1289 \\
    95 & 40 & 4 & 2.5 & 3.5 & 0.9 & 2.15 & 1406 \\
    \bottomrule
  \end{tabular*}
  \label{Ap3}
\end{table*}

\begin{table*}[h!]
  \centering
  \small
  \caption{小区开放前VISSIM正常行驶仿真数据记录表3}
  \begin{tabular*}{\linewidth}{p{50pt}<{\centering}p{50pt}<{\centering}
    p{60pt}<{\centering}p{60pt}<{\centering}p{60pt}<{\centering}p{70pt}<{\centering}}
    \toprule
    数据P.C. & 时间(进入) & 时间(离开) & 车辆编号& 速度(m/s) & 车辆长度(m) \\
    \midrule
    1 & 9.34 & -1 & 4 & 14.7 & 4.76 \\
    1 & -1 & 9.67 & 4 & 14.7 & 4.76 \\
    7 & 19.34 & -1 & 3 & 14.7 & 4.76 \\
    7 & -1 & 19.66 & 3 & 14.8 & 4.76 \\
    6 & 20.35 & -1 & 4 & 14 & 4.76 \\
    6 & -1 & 20.69 & 4 & 14 & 4.76 \\
    1 & 21.49 & -1 & 11 & 14.8 & 4.61 \\
    2 & 21.43 & -1 & 5 & 15.9 & 4.55 \\
    1 & -1 & 21.8 & 11 & 14.9 & 4.61 \\
    2 & -1 & 21.72 & 5 & 15.9 & 4.55 \\
    5 & 22.36 & -1 & 6 & 15.4 & 4.61 \\
    5 & -1 & 22.66 & 6 & 15.4 & 4.61 \\
    2 & 25.81 & -1 & 7 & 13.8 & 4.11 \\
    2 & -1 & 26.11 & 7 & 13.8 & 4.11 \\
    2 & 27.19 & -1 & 8 & 14.1 & 10.21 \\
    2 & -1 & 27.92 & 8 & 14.2 & 10.21 \\
    7 & 29.24 & -1 & 10 & 15.1 & 4.76 \\
    7 & -1 & 29.55 & 10 & 15.1 & 4.76 \\
    6 & 32.23 & -1 & 11 & 14.5 & 4.61 \\
    6 & -1 & 32.55 & 11 & 14.6 & 4.61 \\
    5 & 35.38 & -1 & 12 & 14.8 & 4.55 \\
    5 & -1 & 35.68 & 12 & 14.9 & 4.55 \\
    7 & 36.42 & -1 & 13 & 14.5 & 4.55 \\
    7 & -1 & 36.73 & 13 & 14.5 & 4.55 \\
    1 & 38.87 & -1 & 16 & 15.4 & 4.61 \\
    1 & -1 & 39.16 & 16 & 15.4 & 4.61 \\
    6 & 49.09 & -1 & 16 & 15.1 & 4.61 \\
    6 & -1 & 49.4 & 16 & 15.1 & 4.61 \\
    1 & 49.91 & -1 & 24 & 15.1 & 4.55 \\
    1 & -1 & 50.21 & 24 & 15.2 & 4.55 \\
    5 & 50.59 & -1 & 17 & 15.3 & 4.55 \\
    5 & -1 & 50.89 & 17 & 15.3 & 4.55 \\
    3 & 56.27 & -1 & 14 & 15.8 & 4.76 \\
    3 & -1 & 56.57 & 14 & 15.9 & 4.76 \\
    7 & 57.12 & -1 & 23 & 15.9 & 4.76 \\
    7 & -1 & 57.42 & 23 & 15.8 & 4.76 \\
    6 & 60.56 & -1 & 24 & 14.6 & 4.55 \\
    6 & -1 & 60.87 & 24 & 14.7 & 4.55 \\
    3 & 63.02 & -1 & 18 & 15.1 & 4.11 \\
    3 & -1 & 63.29 & 18 & 15.1 & 4.11 \\
    3 & 66.53 & -1 & 15 & 15 & 4.11 \\
    1 & 66.65 & -1 & 33 & 15 & 4.4 \\
    1 & -1 & 66.94 & 33 & 15.1 & 4.4 \\
    3 & -1 & 66.8 & 15 & 14.9 & 4.11 \\
    5 & 67.35 & -1 & 28 & 14.7 & 4.76 \\
    5 & -1 & 67.68 & 28 & 14.7 & 4.76 \\
    \bottomrule
  \end{tabular*}
  \label{Ap3}
\end{table*}

\newpage

\begin{table*}[h!]
  \centering
  \small
  \caption{小区开放前VISSIM正常行驶仿真数据记录表1}
  \begin{tabular*}{\linewidth}{p{50pt}<{\centering}p{50pt}<{\centering}
    p{60pt}<{\centering}p{60pt}<{\centering}p{60pt}<{\centering}p{70pt}<{\centering}}
    \toprule
    数据P.C. & 时间(进入) & 时间(离开) & 车辆编号& 速度(m/s) & 车辆长度(m) \\
    \midrule
    5 & -1 & 80.44 & 35 & 15 & 4.55 \\
    7 & 84.64 & -1 & 37 & 14.5 & 10.21 \\
    7 & -1 & 85.35 & 37 & 14.4 & 10.21 \\
    3 & 89.77 & -1 & 31 & 14.7 & 4.76 \\
    3 & -1 & 90.09 & 31 & 14.8 & 4.76 \\
    2 & 90.42 & -1 & 32 & 15.2 & 4.11 \\
    2 & -1 & 90.69 & 32 & 15.2 & 4.11 \\
    1 & 90.84 & -1 & 40 & 1.7 & 11.54 \\
    3 & 93.85 & -1 & 34 & 13.9 & 4.76 \\
    3 & -1 & 94.19 & 34 & 14 & 4.76 \\
    1 & -1 & 98.34 & 40 & 1.8 & 11.54 \\
    1 & 101.18 & -1 & 44 & 3.5 & 4.4 \\
    2 & 101.41 & -1 & 29 & 14.6 & 4.61 \\
    2 & -1 & 101.73 & 29 & 14.5 & 4.61 \\
    1 & -1 & 101.8 & 44 & 4.3 & 4.4 \\
    3 & 103.39 & -1 & 36 & 14.5 & 4.76 \\
    3 & -1 & 103.72 & 36 & 1.5 & 4.76 \\
    2 & 109.33 & -1 & 39 & 15.7 & 4.76 \\
    2 & -1 & 109.63 & 39 & 15.7 & 4.76 \\
    1 & 113.78 & -1 & 52 & 3.3 & 4.11 \\
    1 & -1 & 114.75 & 52 & 5.6 & 4.11 \\
    6 & 116.66 & -1 & 46 & 15 & 4.4 \\
    6 & -1 & 116.96 & 46 & 15 & 4.4 \\
    4 & 117.67 & -1 & 1 & 2 & 0.44 \\
    4 & -1 & 117.89 & 1 & 2 & 0.44 \\
    6 & 119.37 & -1 & 44 & 14.2 & 4.4 \\
    6 & -1 & 119.68 & 44 & 14.1 & 4.4 \\
    1 & 127.94 & -1 & 56 & 1.2 & 0.34 \\
    1 & -1 & 128.22 & 56 & 1.2 & 0.34 \\
    3 & 128.54 & -1 & 45 & 12.7 & 4.34 \\
    3 & -1 & 128.87 & 45 & 13 & 4.34 \\
    4 & 128.86 & -1 & 38 & 3.4 & 1.45 \\
    4 & -1 & 129.28 & 38 & 3.5 & 1.45 \\
    6 & 132.61 & -1 & 52 & 14.2 & 4.11 \\
    1 & 132.94 & -1 & 60 & 1.6 & 4.4 \\
    1 & 101.18 & -1 & 44 & 3.5 & 4.4 \\
    2 & 101.41 & -1 & 29 & 14.6 & 4.61 \\
    2 & -1 & 101.73 & 29 & 14.5 & 4.61 \\
    1 & -1 & 101.8 & 44 & 4.3 & 4.4 \\
    3 & 103.39 & -1 & 36 & 14.5 & 4.76 \\
    3 & -1 & 103.72 & 36 & 1.5 & 4.76 \\
    2 & 109.33 & -1 & 39 & 15.7 & 4.76 \\
    2 & -1 & 109.63 & 39 & 15.7 & 4.76 \\
    1 & 113.78 & -1 & 52 & 3.3 & 4.11 \\
    1 & -1 & 114.75 & 52 & 5.6 & 4.11 \\
    6 & 116.66 & -1 & 46 & 15 & 4.4 \\
    \bottomrule
  \end{tabular*}
  \label{Ap4}
\end{table*}



\begin{table*}[h!]
  \centering
  \small
  \caption{小区开放前VISSIM正常行驶仿真数据记录表2}
  \begin{tabular*}{\linewidth}{p{50pt}<{\centering}p{50pt}<{\centering}
    p{60pt}<{\centering}p{60pt}<{\centering}p{60pt}<{\centering}p{70pt}<{\centering}}
    \toprule
    数据P.C. & 时间(进入) & 时间(离开) & 车辆编号& 速度(m/s) & 车辆长度(m) \\
    \midrule
    1 & 127.94 & -1 & 56 & 1.2 & 0.34 \\
    1 & -1 & 128.22 & 56 & 1.2 & 0.34 \\
    3 & 128.54 & -1 & 45 & 12.7 & 4.34 \\
    3 & -1 & 128.87 & 45 & 13 & 4.34 \\
    4 & 128.86 & -1 & 38 & 3.4 & 1.45 \\
    4 & -1 & 129.28 & 38 & 3.5 & 1.45 \\
    2 & -1 & 90.69 & 32 & 15.2 & 4.11 \\
    1 & 90.84 & -1 & 40 & 1.7 & 11.54 \\
    3 & 93.85 & -1 & 34 & 13.9 & 4.76 \\
    3 & -1 & 94.19 & 34 & 14 & 4.76 \\
    1 & -1 & 98.34 & 40 & 1.8 & 11.54 \\
    1 & 101.18 & -1 & 44 & 3.5 & 4.4 \\
    2 & 101.41 & -1 & 29 & 14.6 & 4.61 \\
    2 & -1 & 101.73 & 29 & 14.5 & 4.61 \\
    1 & -1 & 101.8 & 44 & 4.3 & 4.4 \\
    3 & 103.39 & -1 & 36 & 14.5 & 4.76 \\
    3 & -1 & 103.72 & 36 & 1.5 & 4.76 \\
    2 & 109.33 & -1 & 39 & 15.7 & 4.76 \\
    2 & -1 & 109.63 & 39 & 15.7 & 4.76 \\
    1 & 113.78 & -1 & 52 & 3.3 & 4.11 \\
    1 & -1 & 114.75 & 52 & 5.6 & 4.11 \\
    6 & 116.66 & -1 & 46 & 15 & 4.4 \\
    6 & -1 & 116.96 & 46 & 15 & 4.4 \\
    4 & 117.67 & -1 & 1 & 2 & 0.44 \\
    4 & -1 & 117.89 & 1 & 2 & 0.44 \\
    6 & 119.37 & -1 & 44 & 14.2 & 4.4 \\
    6 & -1 & 119.68 & 44 & 14.1 & 4.4 \\
    1 & 127.94 & -1 & 56 & 1.2 & 0.34 \\
    1 & -1 & 128.22 & 56 & 1.2 & 0.34 \\
    3 & 128.54 & -1 & 45 & 12.7 & 4.34 \\
    3 & -1 & 128.87 & 45 & 13 & 4.34 \\
    4 & 128.86 & -1 & 38 & 3.4 & 1.45 \\
    4 & -1 & 129.28 & 38 & 3.5 & 1.45 \\
    6 & 132.61 & -1 & 52 & 14.2 & 4.11 \\
    1 & 132.94 & -1 & 60 & 1.6 & 4.4 \\
    1 & 101.18 & -1 & 44 & 3.5 & 4.4 \\
    2 & 101.41 & -1 & 29 & 14.6 & 4.61 \\
    2 & -1 & 101.73 & 29 & 14.5 & 4.61 \\
    1 & -1 & 101.8 & 44 & 4.3 & 4.4 \\
    3 & 103.39 & -1 & 36 & 14.5 & 4.76 \\
    3 & -1 & 103.72 & 36 & 1.5 & 4.76 \\
    2 & -1 & 109.63 & 39 & 15.7 & 4.76 \\
    1 & 113.78 & -1 & 52 & 3.3 & 4.11 \\
    1 & -1 & 114.75 & 52 & 5.6 & 4.11 \\
    6 & 116.66 & -1 & 46 & 15 & 4.4 \\
    6 & -1 & 116.96 & 46 & 15 & 4.4 \\
    \bottomrule
  \end{tabular*}
  \label{Ap4}
\end{table*}







\end{appendices}
\end{document}
%%
%% This work consists of these files mcmthesis.dtx,
%%                                   figures/ and
%%                                   code/,
%% and the derived files             mcmthesis.cls,
%%                                   mcmthesis-demo.tex,
%%                                   README,
%%                                   LICENSE,
%%                                   mcmthesis.pdf and
%%                                   mcmthesis-demo.pdf.
%%
%% End of file `mcmthesis-demo.tex'.
