

%abstract---------------
{\song\xiaosihao
\setlength{\parindent}{2em}对于问题1我们首先建立了基于成藏思路和题目中给定的资源量与储层参数的线性关系,根据天然气水合物静态赋存特征,从而计算天然气水合物资源量,可视化展现了空间分布,体现了总体分布深度和密度较为不均。


\setlength{\parindent}{2em}
对于问题 2 我们尝试了不同的拟合概率模型,使用二维核密度估计有效厚度,并进一步计算有效厚度,孔隙度,天然气水合物饱和度的变化规律。(你再说说?)

\setlength{\parindent}{2em}对于问题 3 我们使用二维核密度估计和三维克里金插值分析每个井位在Z轴上的资源总量,使用了高斯核函数来平滑数据,从而得到一个连续的概率密度函数。

\setlength{\parindent}{2em}对于问题 4 我们定义了一个双目标优化问题,分别是井间距离最大化和总预测资源量最大化,在X-Y平面上实施了二维克里金插值,使用了球形变异函数和粒子群优化算法,从而计算出最优的新井位置信息。

\setlength{\parindent}{2em}(第 5 段)优化结果及总结:在 · · · · · · 条件下,针对 · · · · · · 模型进行适当修改与优
化,这种条件的改变可能来自你的一种猜想或建议。要注意合理性。此推广模型可以不
深入研究,也可以没有具体结果。}
\begin{rmk}
    目前是450字,第五段还可以写150.
\end{rmk}






