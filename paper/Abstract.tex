

%abstract---------------
{\song\xiaosihao
\setlength{\parindent}{2em}
在响应国家能源安全和可持续发展的战略需求下,本研究致力于天然气水合物(Natural Gas Hydrate/Gas Hydrate),亦称“可燃冰”的深入探索。作为一种在高压低温条件下形成的类冰状结晶物质,天然气水合物在全球能源转型中扮演着至关重要的角色。
本研究将利用14 个位置进行钻孔勘探的井位信息,确定天然气水合物分布范围,变化规律和资源量,并依次指导新井位的开采选址。
\setlength{\parindent}{2em}问题1我们首先建立了基于成藏思路和题目中给定的资源量与储层参数的线性关系,根据天然气水合物静态赋存特征,从而计算天然气水合物资源量,可视化展现了空间分布,体现了总体分布深度和密度较为不均。

\setlength{\parindent}{2em}
问题 2 和问题 3 我们联合解答,首先尝试了不同的拟合概率模型,使用二维核密度估计和三维克里金插值分析每个井位在Z轴上的资源总量,使用了高斯核函数来平滑数据,从而得到一个连续的概率密度函数。
随后进一步计算并得到天然气水合物饱和度的变化规律。

\setlength{\parindent}{2em}问题 4 我们定义了一个双目标优化问题,分别是井间距离最大化和总预测资源量最大化,在X-Y平面上实施了二维克里金插值,使用了球形变异函数和粒子群优化算法,从而计算出最优的新井位置信息。
}
\begin{rmk}

\end{rmk}






