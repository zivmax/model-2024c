

%abstract---------------
{\song\xiaosihao
\setlength{\parindent}{2em}\textcolor{red}{(第1段)	问题重述+简要思想:首先简要叙述所给问题的背景和动机,并分别分析每个小问题的特点(以下以三个问题为例)。根据这些特点说出自己的思想:针对于问题1,采用~$\cdots \cdots$ 的方法解决;针对问题2用~$\cdots \cdots$ 的方法解决;针对问题3用~$\cdots \cdots$ 的方法解决。}


\setlength{\parindent}{2em}\textcolor{red}{(第2段)	模型建立及求解结果:介绍思想和模型: 对于问题1我们首先建立了~$\cdots \cdots$ 模型I。首先利用~$\cdots \cdots$ ,其次计算了~$\cdots \cdots$ ,并借助~$\cdots \cdots$ 数学算法和~$\cdots \cdots$ 软件得出了~$\cdots \cdots$ 结论。}

\setlength{\parindent}{2em} \textcolor{red}{(第3段)	对于问题2我们用~$\cdots \cdots$(模型的建立与求解结果的陈述中,思想、模型、软件和结果必须描述清晰,亮点详细说明需突出。}

\setlength{\parindent}{2em}\textcolor{red}{(第4段)	对于问题3我们用~$\cdots \cdots$ (模型的建立与求解结果的陈述中,思想、模型、软件和结果必须描述清晰,亮点详细说明需突出。}

\setlength{\parindent}{2em}\textcolor{red}{(第5段)	优化结果及总结:在~$\cdots \cdots$ 条件下,针对~$\cdots \cdots$ 模型进行适当修改与优化,这种条件的改变可能来自你的一种猜想或建议。要注意合理性。此推广模型可以不深入研究,也可以没有具体结果。}
}

\begin{rmk}
字数300$\sim $600之间,需控制在一页;摘要中必须将具体方法、模型和所得结果写出来;摘要要求“总分总”,段开头可用“针对问题1,针对问题2,针对问题3..”或者“首先,然后,其次,最后”等词语进行有逻辑的论述。摘要是重中之重,必须严格执行!
\end{rmk}





